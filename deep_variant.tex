\documentclass{beamer}
\usepackage[utf8]{inputenc}

\usepackage{utopia} %font utopia imported

\usetheme{Madrid}
\usecolortheme{default}

%------------------------------------------------------------
%This block of code defines the information to appear in the
%Title page
\title[Deep Variant] %optional
{Deep Variant implementation and algorithms}

\subtitle{A short story}

\author[Raimundo] % (optional)
{F. Raimundo\inst{1}}

\institute[] % (optional)
{
  \inst{1}%
  CEDAR\\
  École Polytechnique
}

\date[] % (optional)
{21th September 2017}

%\logo{\includegraphics[height=1.5cm]{lion-logo.png}}

%End of title page configuration block
%------------------------------------------------------------



%------------------------------------------------------------
%The next block of commands puts the table of contents at the 
%beginning of each section and highlights the current section:

\AtBeginSection[]
{
  \begin{frame}
    \frametitle{Table of Contents}
    \tableofcontents[currentsection]
  \end{frame}
}
%------------------------------------------------------------


\begin{document}

%The next statement creates the title page.
\frame{\titlepage}

\begin{frame}
\frametitle{DeepVariant Paper}

\begin{block}{Original Paper}
"Creating a universal SNP and small indel variant caller with deep neural networks"\\
Ryan Poplin, Dan Newburger, Jojo Dijamco, Nam Nguyen, Dion Loy, Sam Gross, Cory Y. McLean,
Mark A. DePristo\\
DOI: https://doi.org/10.1101/092890
\end{block}

\end{frame}

%---------------------------------------------------------
%This block of code is for the table of contents after
%the title page
\begin{frame}
\frametitle{Table of Contents}
\tableofcontents
\end{frame}
%---------------------------------------------------------

\section{Note}
\begin{frame}
    \begin{itemize}
     \item not sensible to alignment method
     \item needs high sensitity and low specificity
     \item can work without realignment
     \item can work on different sequencers (using provided BAM)
     \item no need for VQSR
     \item works accross species and can be trained on species with more data (human and mice)
     \item works accross sequencers (need ground truth but that's all)
     \item opens door to transfer learning
    \end{itemize}
\end{frame}

\section{Motivation}

\begin{frame}
    \frametitle{FDA precision challenge}

    Objectives
    \begin{itemize}
        \item Created with the Genome in a Bottle Consortium.
        \item Only one dataset with high quality annotations available (NA12878).
        \item Quality tested on new datasets.
        \item Evaluation of FScore, recall and precision for SNPs and Indels.
    \end{itemize}

\end{frame}

\begin{frame}
    \frametitle{Deep Variant results}

    \begin{itemize}
        \item Best FScore for SNPs, honorable mention for precision and recall.
        \item First method to use Deep Learning (DL).
        \item Proof of concept that DL is a promising method.
    \end{itemize}
\end{frame}

\section{Preprocessing}

%---------------------------------------------------------
%Changing visivility of the text
\begin{frame}
    \frametitle{Haplotype-aware realignment of reads}

    \begin{itemize}
        \item Reads are previously mapped (method unspecified).
        \item Candidates windows (size unspecified) are chosen based on mismatches and soft clips.
        \item Creation of De-Bruijn graphs for kmers of size 20 to 75 (increment of 5) for the
              reference and all overlapping reads in the window.
        \item Edges are weighted according to their number of occurences.
    \end{itemize}
\end{frame}

\begin{frame}
    \frametitle{Haplotype-aware realignment of reads (cont)}

    \begin{itemize}
        \item Edges with weight ower than 3 are trimmed (except for reference).
        \item Candidates haplotype are selected by traversing the graph, the two most likel are
              selected (evaluated with HMM).
        \item Reads are realigned with Smith-Waterson with affine gap penality.
        \item Position and CIGAR strings are updated in the reads.
    \end{itemize}
\end{frame}

\begin{frame}
    \frametitle{Finding candidate variants}

    \begin{itemize}
        \item Each position in the genome is evaluated.
        \item Collect all reads overlaping that position and aligned.
        \item Each possible allele is considered.
        \item If it is not reference, is present at least a number of time and represents a certain
            fraction of alleles it is emitted as a candidate.
    \end{itemize}

\end{frame}

\begin{frame}
    \frametitle{Comparative with GATK}

    \begin{itemize}
        \item No VQSR.
        \item No first pass of HaplotypeCaller.
        \item Mark duplicates used, but not described.
    \end{itemize}

\end{frame}

\begin{frame}
    \frametitle{Prepocessing conclusion}

    \begin{itemize}
        \item The realignment can be skipped (with lower results, only done for not illumina).
        \item The candidates are emitted with high sensitivity and low specificity on purpose.
        \item This whole step can be skipped (by using provided candidates).
    \end{itemize}

\end{frame}

\section{Transformation to image}

%---------------------------------------------------------
%Highlighting text
\begin{frame}
    \frametitle{Property of the image}

    \begin{itemize}
        \item An 221x100px image is created for each candidate variant.
        \item First 5 rows are for the reference genome.
        \item Each row below is used for an overlapping read.
        \item Each column encodes for the base pair at that position (relative to the ref) in the
            row of the read.
        \item Reads are thus 221bp long and there is at most 95 reads.
        \item Center column is assumed (by me) to be the position of the candidate.
    \end{itemize}

\end{frame}

\begin{frame}
    \frametitle{Pixel encoding}

    \begin{itemize}
        \item Red: encodes the base color (A: 250, G: 180, T: 100, C: 30)
        \item Green: encodes the quality (intensity linear in the quality). 
        \item Blue: direction of the strand (70 if positive, 240 otherwise).
        \item Alpha: encodes if the read is equal to the ref and if there is an alternative allele.
    \end{itemize}

\end{frame}
%---------------------------------------------------------

\section{Classifier: Training and Evaluation}

\begin{frame}
    \frametitle{Problem statement}

    \begin{itemize}
        \item Supervised.
        \item Classification into "hom-ref", "het", "hom-alt".
        \item Ground truth comes from NA12878.
        \item Trained on chr1-18, hyperparam tuned on chr19, test on chr20-22.
    \end{itemize}
\end{frame}

\begin{frame}
    \frametitle{Neural architecture}

    \begin{itemize}
        \item Inception v2.
        \item Pretrained on ImageNet.
    \end{itemize}
\end{frame}

\begin{frame}
    \frametitle{Training}

    \begin{itemize}
        \item 9 partitions.
        \item Last layer initialised with gaussian random weights.
        \item SGD with 32 images per batch and 8 replicates.
        \item Training stoped after 80 hours or 250.000 epochs or training accuracy convergence.
    \end{itemize}
\end{frame}

\begin{frame}
    \frametitle{Evaluation against GATK}

    \begin{itemize}
        \item GATK implemented following Best practices and VQSR for all chr.
        \item GATK implemented following Best practices and VQSR for chr1-18.
        \item Beats both.
    \end{itemize}
\end{frame}

\begin{frame}
    \frametitle{Evaluation on Mice}

    \begin{itemize}
        \item Used MGP data.
        \item Beats state of the art (F1: 98.29\% vs 97.84\%).
        \item Gets better results when trained on NA12878 than mouse genome.
    \end{itemize}
\end{frame}

\begin{frame}
    \frametitle{Evaluation on other sequencers}

    \begin{itemize}
        \item Used MGP data.
        \item Beats state of the art (F1: 98.29\% vs 97.84\%).
        \item Gets better results when trained on NA12878 than mouse genome.
    \end{itemize}
\end{frame}

\section{Promises of DeepVariant}

\section{Annex: Inception v2 architecture}

\end{document}
